% Options for packages loaded elsewhere
\PassOptionsToPackage{unicode}{hyperref}
\PassOptionsToPackage{hyphens}{url}
%
\documentclass[
]{article}
\usepackage{amsmath,amssymb}
\usepackage{lmodern}
\usepackage{iftex}
\ifPDFTeX
  \usepackage[T1]{fontenc}
  \usepackage[utf8]{inputenc}
  \usepackage{textcomp} % provide euro and other symbols
\else % if luatex or xetex
  \usepackage{unicode-math}
  \defaultfontfeatures{Scale=MatchLowercase}
  \defaultfontfeatures[\rmfamily]{Ligatures=TeX,Scale=1}
\fi
% Use upquote if available, for straight quotes in verbatim environments
\IfFileExists{upquote.sty}{\usepackage{upquote}}{}
\IfFileExists{microtype.sty}{% use microtype if available
  \usepackage[]{microtype}
  \UseMicrotypeSet[protrusion]{basicmath} % disable protrusion for tt fonts
}{}
\makeatletter
\@ifundefined{KOMAClassName}{% if non-KOMA class
  \IfFileExists{parskip.sty}{%
    \usepackage{parskip}
  }{% else
    \setlength{\parindent}{0pt}
    \setlength{\parskip}{6pt plus 2pt minus 1pt}}
}{% if KOMA class
  \KOMAoptions{parskip=half}}
\makeatother
\usepackage{xcolor}
\IfFileExists{xurl.sty}{\usepackage{xurl}}{} % add URL line breaks if available
\IfFileExists{bookmark.sty}{\usepackage{bookmark}}{\usepackage{hyperref}}
\hypersetup{
  hidelinks,
  pdfcreator={LaTeX via pandoc}}
\urlstyle{same} % disable monospaced font for URLs
\usepackage{longtable,booktabs,array}
\usepackage{calc} % for calculating minipage widths
% Correct order of tables after \paragraph or \subparagraph
\usepackage{etoolbox}
\makeatletter
\patchcmd\longtable{\par}{\if@noskipsec\mbox{}\fi\par}{}{}
\makeatother
% Allow footnotes in longtable head/foot
\IfFileExists{footnotehyper.sty}{\usepackage{footnotehyper}}{\usepackage{footnote}}
\makesavenoteenv{longtable}
\usepackage{graphicx}
\makeatletter
\def\maxwidth{\ifdim\Gin@nat@width>\linewidth\linewidth\else\Gin@nat@width\fi}
\def\maxheight{\ifdim\Gin@nat@height>\textheight\textheight\else\Gin@nat@height\fi}
\makeatother
% Scale images if necessary, so that they will not overflow the page
% margins by default, and it is still possible to overwrite the defaults
% using explicit options in \includegraphics[width, height, ...]{}
\setkeys{Gin}{width=\maxwidth,height=\maxheight,keepaspectratio}
% Set default figure placement to htbp
\makeatletter
\def\fps@figure{htbp}
\makeatother
\usepackage[normalem]{ulem}
% Avoid problems with \sout in headers with hyperref
\pdfstringdefDisableCommands{\renewcommand{\sout}{}}
\setlength{\emergencystretch}{3em} % prevent overfull lines
\providecommand{\tightlist}{%
  \setlength{\itemsep}{0pt}\setlength{\parskip}{0pt}}
\setcounter{secnumdepth}{-\maxdimen} % remove section numbering
\ifLuaTeX
  \usepackage{selnolig}  % disable illegal ligatures
\fi

\author{}
\date{}

\begin{document}

\hypertarget{nezzal-abdelmalek}{%
\section{Nezzal Abdelmalek}\label{nezzal-abdelmalek}}

Nezzal Abdelmalek is an occupational health professor
\texttt{at\ the\ faculty\ of\ medecine\ in\ Badji\ Mokhtar\ University}.

Nezzal has teached to tens of physicians.

\begin{quote}
Nezzal was born on the 17th of october 1950 in Ain Zatout, in Algeria.
\end{quote}

\begin{center}\rule{0.5\linewidth}{0.5pt}\end{center}

\hypertarget{table-of-contents}{%
\subsection{Table of contents}\label{table-of-contents}}

\begin{itemize}
\tightlist
\item
  \protect\hyperlink{at-a-glance}{At a glance}
\item
  \protect\hyperlink{new-york-laboratories}{Annaba}

  \begin{itemize}
  \tightlist
  \item
    \protect\hyperlink{tesla-coil}{Tesla coil}
  \item
    \protect\hyperlink{wireless-lighting}{Wireless lighting}
  \item
    \protect\hyperlink{radio-remote-control}{Radio remote control}\\
  \end{itemize}
\item
  \protect\hyperlink{patents}{Patents}\\
\item
  \protect\hyperlink{legacy-and-honors}{Legacy and honors}\\
\item
  \protect\hyperlink{credits}{Credits}
\end{itemize}

\begin{verbatim}
Disclaimer: This site is used to demonstrate Markdown and is not intended to be a comprehensive outline of Nikola Tesla's life. 
\end{verbatim}

\begin{center}\rule{0.5\linewidth}{0.5pt}\end{center}

\hypertarget{at-a-glance}{%
\subsection{At a glance}\label{at-a-glance}}

\begin{longtable}[]{@{}lr@{}}
\toprule
Key & Value \\
\midrule
\endhead
Born & 10 July 1856, \emph{Modern-day Croatia} \\
Died & 7 January 1943 (aged 86), \emph{New York City, United States} \\
Resting place & Nikola Tesla Museum, \emph{Belgrade, Serbia} \\
Citizenship & Austrian \emph{(1856--1891)}, American
\emph{(1891--1943)} \\
Best known for & Modern alternating current (AC) electricity supply
system \\
\bottomrule
\end{longtable}

\hypertarget{new-york-laboratories}{%
\subsection{New York laboratories}\label{new-york-laboratories}}

\hypertarget{tesla-coil}{%
\subsubsection{Tesla coil}\label{tesla-coil}}

Tesla found the discoveries of \textbf{Heinrich Hertz} refreshing and
decided to repeat the experiment but found that the high frequency
current overheated the iron core and melted the insulation between the
primary and secondary windings in the coil. To fix this problem Tesla
came up with his Tesla coil with an air gap instead of insulating
material between the primary and secondary windings and an iron core
that could be moved to different positions in or out of the coil.

\hypertarget{wireless-lighting}{%
\subsubsection{Wireless lighting}\label{wireless-lighting}}

Tesla attempted to develop a wireless lighting system based on
near-field inductive and capacitive coupling and conducted a series of
public demonstrations where he lit Geissler tubes and even incandescent
light bulbs from across a stage.

\hypertarget{radio-remote-control}{%
\subsubsection{Radio remote control}\label{radio-remote-control}}

Tesla demonstrated a boat that used a coherer-based radio
control---which he dubbed ``\emph{telautomaton}''---to the public during
an electrical exhibition at Madison Square Garden. He tried to sell his
idea to the U.S. military as a type of radio-controlled torpedo, but
they showed little interest until World War I and afterward, when a
number of countries used it in military progra

\hypertarget{patents}{%
\subsection{Patents}\label{patents}}

Tesla obtained around \sout{280} 300 patents worldwide for his
inventions. Many inventions developed by Tesla were not put into patent
protection.

\hypertarget{legacy-and-honors}{%
\subsection{Legacy and honors}\label{legacy-and-honors}}

\begin{itemize}
\tightlist
\item
  \textbf{Places}

  \begin{enumerate}
  \def\labelenumi{\arabic{enumi}.}
  \tightlist
  \item
    Belgrade Nikola Tesla Airport
  \item
    Nikola Tesla Museum Archive in Belgrade
  \item
    TPP Nikola Tesla, the largest power plant in Serbia
  \item
    128 streets in Croatia had been named after Nikola Tesla as of
    November 2008, making him the eighth most common street name origin
    in the country
  \item
    Tesla, a 26 kilometer-wide crater on the far side of the moon
  \item
    2244 Tesla, a minor planet
  \end{enumerate}
\item
  \textbf{Holidays and events}

  \begin{enumerate}
  \def\labelenumi{\arabic{enumi}.}
  \tightlist
  \item
    Day of Science, Serbia, 10 July
  \item
    Day of Nikola Tesla, Association of Teachers in Vojvodina, 4--10
    July
  \item
    Day of Nikola Tesla, Niagara Falls, 10 July
  \item
    Nikola Tesla Day in Croatia, 10 July
  \item
    Nikola Tesla annual electric vehicle rally in Croatia
  \end{enumerate}
\item
  \textbf{Memorials}

  \begin{enumerate}
  \def\labelenumi{\arabic{enumi}.}
  \tightlist
  \item
    The Nikola Tesla Memorial Centre in Smiljan, Croatia, opened in
    2006.
  \end{enumerate}
\end{itemize}

\begin{figure}
\centering
\includegraphics{./assets/img/Muzej_Nikole_Tesle.jpg}
\caption{The Nikola Tesla Memorial Centre in Smiljan, Croatia}
\end{figure}

\hypertarget{credits}{%
\subsection{Credits}\label{credits}}

\begin{itemize}
\tightlist
\item
  Content paraphrased from
  \href{https://en.wikipedia.org/wiki/Nikola_Tesla}{Nikola Tesla
  Wikipedia page}.\\
\item
  Nikola Tesla picture sourced from
  \href{https://commons.wikimedia.org/wiki/File:N.Tesla.JPG}{Wikimedia}.\\
\item
  Nikola Tesla Museum image courtesy
  \href{https://commons.wikimedia.org/w/index.php?curid=3090675}{By
  sr:Корисник:JustUser - sr:wiki}
\end{itemize}

\end{document}
